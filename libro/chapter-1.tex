\chapter{Prólogo}
Este es un curso sobre los Fundamentos de GNU/Linux. En él aprenderás el funcionamiento general del sistema operativo, además del proceso de instalación de éste y el uso de la terminal, una herramienta muy importante para Linux.

El objetivo de este curso no es dominar Linux ni mucho menos. Este curso no intenta ser demasiado teórico ni técnico, la intención es proporcionar herramientas que faciliten al interesado o interesada poder aprender a usar este sistema operativo de forma autodidacta. Para ello, este curso tendrá ejercicios prácticos sobre la terminal para poder aplicar la teoría que se impartirá en estos apuntes. Estos ejercicios se pueden consultar en el repositorio del curso.

Es muy importante recalcar que no entender todo a la primera es normal. A pesar de ser un curso sobre los fundamentos, quizá sea necesario releer apartados y repetir ejercicios lo necesario. Recuerda tomar descansos y preguntar las dudas que necesites resolver. En el \href{https://discord.gg/2qPvfCxD9U}{servidor de Discord de 0xDECODE} estaremos encantados de ayudar.

El autor te anima a que le des una oportunidad a este sistema operativo. Por muy complicado que parezca desde fuera, GNU/Linux es un sistema operativo muy intuitivo y fácil de aprender si te lo propones. Aprenderlo no tiene pérdida, y podría serte de mucha ayuda en un futuro o incluso en la actualidad. No te abrumes si no consigues interiorizar todo el contenido del curso; como he dicho, la intención no es esa, ya que ningún usuario de Linux conoce el sistema de forma completa. Si consigues aprender lo fundamental sobre el funcionamiento de Linux, hasta el punto de poder usarlo de forma cotidiana, este curso habrá cumplido su objetivo.


\textbf{¡Mucha suerte!}